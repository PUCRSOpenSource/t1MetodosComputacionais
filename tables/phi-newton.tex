\begin{table}[H]
\centering 
\begin{tabular}{|c|c|c|c|}
\hline 
Iteração & Aproximação & PHI & Erro \\ 
\hline 
1 & 1.27755102040816e+01 &  1.61803398874989e+00 & 2.40000000000000e+01 \\ 
\hline
2 & 6.68866948275569e+00 &  1.61803398874989e+00 & 6.08684072132594e+00 \\ 
\hline
3 & 3.69532575434784e+00 &  1.61803398874989e+00 & 2.99334372840785e+00 \\ 
\hline
4 & 2.29326108782571e+00 &  1.61803398874989e+00 & 1.40206466652214e+00 \\ 
\hline
5 & 1.74515759568633e+00 &  1.61803398874989e+00 & 5.48103492139371e-01 \\ 
\hline
6 & 1.62452329239167e+00 &  1.61803398874989e+00 & 1.20634303294670e-01 \\ 
\hline
7 & 1.61805271271143e+00 &  1.61803398874989e+00 & 6.47057968023024e-03 \\ 
\hline
8 & 1.61803398890668e+00 &  1.61803398874989e+00 & 1.87238047555383e-05 \\ 
\hline
9 & 1.61803398874989e+00 &  1.61803398874989e+00 & 1.56784363269935e-10 \\ 
\hline
10 & 1.61803398874989e+00 &  1.61803398874989e+00 & 0.00000000000000e+00 \\ 
\hline
\end{tabular}
\caption{Convergência do número de ouro pelo método de Newton}
\label{table:phi-newton}
\end{table}