\begin{table}[H]
\centering 
\begin{tabular}{|c|c|c|c|}
\hline 
Iteração & Aproximação & PHI & Erro \\ 
\hline 
1 & 5.31578947368421e+00 &  1.61803398874989e+00 & 9.00000000000000e+00 \\ 
\hline
2 & 3.03767615760713e+00 &  1.61803398874989e+00 & 2.27811331607708e+00 \\ 
\hline
3 & 2.01512639976464e+00 &  1.61803398874989e+00 & 1.02254975784250e+00 \\ 
\hline
4 & 1.67007003765974e+00 &  1.61803398874989e+00 & 3.45056362104896e-01 \\ 
\hline
5 & 1.61919107776978e+00 &  1.61803398874989e+00 & 5.08789598899588e-02 \\ 
\hline
6 & 1.61803458688502e+00 &  1.61803398874989e+00 & 1.15649088475700e-03 \\ 
\hline
7 & 1.61803398875005e+00 &  1.61803398874989e+00 & 5.98134969331809e-07 \\ 
\hline
8 & 1.61803398874989e+00 &  1.61803398874989e+00 & 1.59872115546023e-13 \\ 
\hline
9 & 1.61803398874989e+00 &  1.61803398874989e+00 & 0.00000000000000e+00 \\ 
\hline
\end{tabular}
\caption{Convergência do número de ouro pelo método de Newton}
\label{table:phi-newton}
\end{table}