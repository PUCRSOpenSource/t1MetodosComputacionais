\begin{table}[H]
\centering 
\begin{tabular}{|c|c|c|c|}
\hline 
Iteração & Aproximação &  PHI & Erro \\ 
\hline 
1 & 1.00000000000000 &  1.61803398874989 & 0.61803398874989 \\ 
\hline
2 & 2.00000000000000 &  1.61803398874989 & 0.38196601125011 \\ 
\hline
3 & 1.50000000000000 &  1.61803398874989 & 0.11803398874989 \\ 
\hline
4 & 1.66666666666667 &  1.61803398874989 & 0.04863267791677 \\ 
\hline
5 & 1.60000000000000 &  1.61803398874989 & 0.01803398874989 \\ 
\hline
6 & 1.62500000000000 &  1.61803398874989 & 0.00696601125011 \\ 
\hline
7 & 1.61538461538462 &  1.61803398874989 & 0.00264937336528 \\ 
\hline
8 & 1.61904761904762 &  1.61803398874989 & 0.00101363029772 \\ 
\hline
9 & 1.61764705882353 &  1.61803398874989 & 0.00038692992637 \\ 
\hline
10 & 1.61818181818182 &  1.61803398874989 & 0.00014782943192 \\ 
\hline
\end{tabular}
\label{table:phi-frac}
\caption{Convergência do número de ouro pelo método de frações continuadas}
\end{table}