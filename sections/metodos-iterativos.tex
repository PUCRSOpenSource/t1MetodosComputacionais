\section{Métodos Iterativos}
\label{sec:metodos-iterativos}

	Um \textbf{método iterativo} é um procedimento matemático que gera uma
	sequência de aproximações, e quando esta sequência é convergente esta
	aproximação pode ser aceita como a solução para uma classe de problemas. Um
	método iterativo é considerado \textbf{convergente} se a sequência de
	valores geradas converge para uma dada aproximação inicial.

	Para uma definição mais teórica, o seguinte autor define:

	\begin{quotation}

		``Um método iterativo consiste em repetir uma determinada operação um
		certo número de vezes até que nos seja fornecida uma aproximação, que
		satisfaça as condições do problema e, para tal, a sequência de valores
		deve ser convergente.''\cite{batista2014metodos}

	\end{quotation}

	Nos problemas do tipo encontre a raíz da equação, um método iterativo usa
	uma suposição inicial para gerar sucessivas aproximações à uma solução. Em
	Contraste, \textbf{métodos diretos} tentam resolver o problema em uma
	sequência \emph{finita} de operações.

	Um método iterativo é formado por quatro partes:~\cite{claudio2000calculo}

	\begin{enumerate}[a)]

		\item Estimativa inicial: uma ou mais aproximações para a raiz desejada.

		\item Atualização: uma fórmula que atualize a solução aproximada.

		\item Critério de parada: uma forma de estabelecer quando parar o
			processo iterativo em qualquer caso.

		\item Estimador de exatidão: está associado ao critério de parada e
			provê uma estimativa do erro cometido.

	\end{enumerate}

	Nas próximas seções serão apresentadas as constantes utilizando métodos
	iterativos, bem como a análise de convergência. A análise de convergência é
	obtida calculando a diferença entre a constate e o valor obtido na iteração,
	ou seja, usamos a medida de erro absoluto. Os algoritmos utilizados seguem o
	mesmo formato, no sentido de que sempre que o executamos é informado o
	número máximo de iterações e o erro tolerável. Quando uma das condições é
	satisfeita, assim começamos a análise dos resultados obtidos.

	Os algoritmos utilizados para a extração dos resultados foram escritos e
	executados no ambiente matemático \emph{octave}, pois o grupo possui uma
	prévia familiaridade com a ferramenta. 
	
	\subsection{Número de Ouro}
		
		O número de ouro, também denotada pela letra grega $\phi$ e é obitdo a
		partir da raiz positiva da Equação~\ref{eq:phi}

		\begin{equation}
			x^2-x-1
		\label{eq:phi}
		\end{equation}

		Existem algums algoritmos para realizar a aproximação neste artigo,
		sendo duas destas formas  abordadas. A primeira utilizando frações
		continuas e o segundo calculando as raizaes da equação utiizando método
		de newton.

		\subsubsection{Frações Continuadas}

			Frações conitinuadas são formas de representar números reais de tal
			forma que a expressão básica tem o seguinte formato:

			\begin{equation}
			\label{eq:phi-frac}
				a_0 + \frac{b_0}{a_1 + \frac{b_1}{a_n + \dots}}
			\end{equation}

			Para calcular o $\phi$ devemos substituir \emph{a} e \emph{b} por
			\emph{1} na Fração~\ref{eq:phi}. A Tabela~\ref{table:phi-frac} foi
			obtida utlizando o método de frações continuadas de forma que o
			número máximo de iterações é 50 e o erro aceitável é $10^{-15}$. Onde
			podemos observar que a partir da iteração \emph{39} o erro foi
			menor do que a tolerancia.

			\begin{table}[H]
\centering 
\begin{tabular}{|c|c|c|c|}
\hline 
Iteração & Aproximação &  PHI & Erro \\ 
\hline 
1 & 1.00000000000000 &  1.61803398874989 & 0.61803398874989 \\ 
\hline
2 & 2.00000000000000 &  1.61803398874989 & 0.38196601125011 \\ 
\hline
3 & 1.50000000000000 &  1.61803398874989 & 0.11803398874989 \\ 
\hline
4 & 1.66666666666667 &  1.61803398874989 & 0.04863267791677 \\ 
\hline
5 & 1.60000000000000 &  1.61803398874989 & 0.01803398874989 \\ 
\hline
6 & 1.62500000000000 &  1.61803398874989 & 0.00696601125011 \\ 
\hline
7 & 1.61538461538462 &  1.61803398874989 & 0.00264937336528 \\ 
\hline
8 & 1.61904761904762 &  1.61803398874989 & 0.00101363029772 \\ 
\hline
9 & 1.61764705882353 &  1.61803398874989 & 0.00038692992637 \\ 
\hline
10 & 1.61818181818182 &  1.61803398874989 & 0.00014782943192 \\ 
\hline
11 & 1.61797752808989 &  1.61803398874989 & 0.00005646066001 \\ 
\hline
12 & 1.61805555555556 &  1.61803398874989 & 0.00002156680566 \\ 
\hline
13 & 1.61802575107296 &  1.61803398874989 & 0.00000823767693 \\ 
\hline
14 & 1.61803713527851 &  1.61803398874989 & 0.00000314652862 \\ 
\hline
15 & 1.61803278688525 &  1.61803398874989 & 0.00000120186465 \\ 
\hline
16 & 1.61803444782168 &  1.61803398874989 & 0.00000045907179 \\ 
\hline
17 & 1.61803381340013 &  1.61803398874989 & 0.00000017534977 \\ 
\hline
18 & 1.61803405572755 &  1.61803398874989 & 0.00000006697766 \\ 
\hline
19 & 1.61803396316671 &  1.61803398874989 & 0.00000002558319 \\ 
\hline
20 & 1.61803399852180 &  1.61803398874989 & 0.00000000977191 \\ 
\hline
21 & 1.61803398501736 &  1.61803398874989 & 0.00000000373254 \\ 
\hline
22 & 1.61803399017560 &  1.61803398874989 & 0.00000000142570 \\ 
\hline
23 & 1.61803398820533 &  1.61803398874989 & 0.00000000054457 \\ 
\hline
24 & 1.61803398895790 &  1.61803398874989 & 0.00000000020801 \\ 
\hline
25 & 1.61803398867044 &  1.61803398874989 & 0.00000000007945 \\ 
\hline
26 & 1.61803398878024 &  1.61803398874989 & 0.00000000003035 \\ 
\hline
27 & 1.61803398873830 &  1.61803398874989 & 0.00000000001159 \\ 
\hline
28 & 1.61803398875432 &  1.61803398874989 & 0.00000000000443 \\ 
\hline
29 & 1.61803398874820 &  1.61803398874989 & 0.00000000000169 \\ 
\hline
30 & 1.61803398875054 &  1.61803398874989 & 0.00000000000065 \\ 
\hline
31 & 1.61803398874965 &  1.61803398874989 & 0.00000000000025 \\ 
\hline
32 & 1.61803398874999 &  1.61803398874989 & 0.00000000000009 \\ 
\hline
33 & 1.61803398874986 &  1.61803398874989 & 0.00000000000004 \\ 
\hline
34 & 1.61803398874991 &  1.61803398874989 & 0.00000000000001 \\ 
\hline
35 & 1.61803398874989 &  1.61803398874989 & 0.00000000000001 \\ 
\hline
36 & 1.61803398874990 &  1.61803398874989 & 0.00000000000000 \\ 
\hline
37 & 1.61803398874989 &  1.61803398874989 & 0.00000000000000 \\ 
\hline
\end{tabular}
\label{table:phi-frac}
\caption{Convergência do número de ouro pelo método de frações continuadas}
\end{table}


			\begin{figure}[H]
				\centering
				\includegraphics[width=100mm]{golden_frac.png}
				\caption{Convergência do número de ouro pelo método de frações continuadas}
				\label{golden_frac}
			\end{figure}


			Os valores foram obtidos a partir do algoritmo abaixo:

			\begin{lstlisting}

			function phi_frac(iteration, err)
			phi = (1 + sqrt(5))/2
			aux = 1
			x(1,1) = 1
			x(2,1) = aux
			x(3,1) = phi
			x(4,1) = abs(phi - aux)

			i = 2
			while xor(i <= iteration,  err > x(4, i-1))
				aux = double(1 + 1/aux)
				x(1,i) = i
				x(2,i) = aux
				x(3,i) = phi
				x(4,i) = abs(phi - aux)
				i++
			end

			end

			\end{lstlisting}

		\subsubsection{Método de Newton}

			A ideia do método de Newton é a partir de um valor inicial
			arbitrário, então a função é aproximada por sua reta tangente. O $x$
			que intercepta a reta e a função é computado, e ele será uma melhor
			aproximação que o chute inicial. O método, então, pode ser iterado.

			Analizando novamente o número de ouro, mas com o método de newton,
			um número muito menor de iterações é observado.

			\begin{table}[H]
\centering 
\begin{tabular}{|c|c|c|c|}
\hline 
Iteração & Aproximação & PHI & Erro \\ 
\hline 
1 & 1.52711864406780e+01 &  1.61803398874989e+00 & 2.90000000000000e+01 \\ 
\hline
\end{tabular}
\caption{Convergência do número de ouro pelo método de Newton}
\label{table:phi-newton}
\end{table}

			Isso pode ser observado no gráfico abaixo.

			\begin{figure}[H]
				\centering
				\includegraphics[width=100mm]{golden_newton.png}
				\caption{Convergência do número de ouro pelo método de newton}
				\label{golden_newton}
			\end{figure}

			O método iterativo utilizado foi descrito pelo seguinte código:

			\begin{lstlisting}

			function [x, ex] = newton( f, df, x0, tol, nmax)
				f = inline(f);
				df = inline(df);
				x(1) = double(x0 - (f(x0)/df(x0)));
				ex(1) = abs(x(1)-x0);
				k = 2;
				while k <= nmax && ex(k-1) > tol
					 x(k) = double(double(x(k-1)) - double((f(x(k-1))/df(x(k-1)))));
					 ex(k) = abs(x(k)-x(k-1));
					 k = k+1;
				end
			end

			\end{lstlisting}

	\subsection{Pi($\pi$)}

		\subsubsection{Método Utilizando Funções Trigonométricas}

			Encontramos em um
			\href{http://mathforum.org/library/drmath/view/65244.html}{fórum de
			matemática} um método iterativo que calcula $ \pi $ de uma forma
			aparentemente mais simples, apesar de sua complexidade estar
			escondida na função \emph{sin}. O método está descrito a seguir:

			\begin{equation}
			\label{magic_equation}
				P(n+1) = P(n) + sin(P(n))
			\end{equation}

			$P(n)$ seria a aproximação de $\pi$ na iteração $n$. Esse método
			consegue convergir para $\pi$ com um número muito baixo de
			iterações.

			\begin{table}[H]
\centering 
\begin{tabular}{|c|c|c|c|}
\hline 
Iteração & Aproximação & pi & Erro \\ 
\hline 
1 & 3.14112000805987e+00 &  3.14159265358979e+00 & 4.72645529925764e-04 \\ 
\hline
2 & 1.84147098480790e+00 &  3.14159265358979e+00 & 2.14159265358979e+00 \\ 
\hline
3 & 3.14159265357220e+00 &  3.14159265358979e+00 & 4.72645529925764e-04 \\ 
\hline
4 & 3.14159265358979e+00 &  3.14159265358979e+00 & 0.00000000000000e+00 \\ 
\hline
\end{tabular}
\caption{Convergência de pi utilizando funções trigonométricas}
\label{table:pi-sin}
\end{table}

			O seguinte gráfico foi gerado com a análise dos resultos:

			\begin{figure}[H]
				\centering
				\includegraphics[width=100mm]{pi_magic.png}
				\caption{Convergência do $\pi$}
				\label{fig:pi-magic}
			\end{figure}

			O método iterativo utilizado foi descrito pelo seguinte código:

			\begin{lstlisting}

			function [pif, pi_vec] = pi_it(iteration)
				pif(1) = 3 + sin(3);
				pi_vec(1) = pi
				for i = 2:iteration
					pif(i) = pif(i-1) + sin(pif(i-1));
					pi_vec(i) = pi;
					aux = pif(i);
				end
			end

			\end{lstlisting}

	\subsection{$e^x$}

		A função $e^x$ é um função exponencial cuja base é o número de Euler,
		também conhecida como função exponencial natural.

		Para o calculo da da função utilizamos a seguinte série de Taylor
		
		\begin{equation}
			\sum_{n=1}^{n} = \frac{x^n}{n!}
		\end{equation}

		Para exemplificar a foram realizadas 40 iteração para a seguinte função
		$e^5$, e obtivemos os seguintes resultaodos:

		\begin{table}[H]
\centering 
\begin{tabular}{|c|c|c|c|}
\hline 
Iteração & Aproximação & $e^x$ & Erro \\ 
\hline 
1 & 1.00000000000000e+00 &  1.48413159102577e+02 & 1.47413159102577e+02 \\ 
\hline
2 & 6.00000000000000e+00 &  1.48413159102577e+02 & 1.47413159102577e+02 \\ 
\hline
3 & 1.85000000000000e+01 &  1.48413159102577e+02 & 1.42413159102577e+02 \\ 
\hline
4 & 3.93333333333333e+01 &  1.48413159102577e+02 & 1.29913159102577e+02 \\ 
\hline
5 & 6.53750000000000e+01 &  1.48413159102577e+02 & 1.09079825769243e+02 \\ 
\hline
6 & 9.14166666666667e+01 &  1.48413159102577e+02 & 8.30381591025766e+01 \\ 
\hline
7 & 1.13118055555556e+02 &  1.48413159102577e+02 & 5.69964924359099e+01 \\ 
\hline
8 & 1.28619047619048e+02 &  1.48413159102577e+02 & 3.52951035470210e+01 \\ 
\hline
9 & 1.38307167658730e+02 &  1.48413159102577e+02 & 1.97941114835290e+01 \\ 
\hline
10 & 1.43689456569665e+02 &  1.48413159102577e+02 & 1.01059914438464e+01 \\ 
\hline
11 & 1.46380601025132e+02 &  1.48413159102577e+02 & 4.72370253291169e+00 \\ 
\hline
12 & 1.47603848504890e+02 &  1.48413159102577e+02 & 2.03255807744432e+00 \\ 
\hline
13 & 1.48113534954789e+02 &  1.48413159102577e+02 & 8.09310597686419e-01 \\ 
\hline
14 & 1.48309568204750e+02 &  1.48413159102577e+02 & 2.99624147787284e-01 \\ 
\hline
15 & 1.48379580079737e+02 &  1.48413159102577e+02 & 1.03590897826081e-01 \\ 
\hline
16 & 1.48402917371399e+02 &  1.48413159102577e+02 & 3.35790228399446e-02 \\ 
\hline
17 & 1.48410210275043e+02 &  1.48413159102577e+02 & 1.02417311778993e-02 \\ 
\hline
18 & 1.48412355246703e+02 &  1.48413159102577e+02 & 2.94882753351544e-03 \\ 
\hline
19 & 1.48412951072164e+02 &  1.48413159102577e+02 & 8.03855873414250e-04 \\ 
\hline
20 & 1.48413107868338e+02 &  1.48413159102577e+02 & 2.08030412267135e-04 \\ 
\hline
21 & 1.48413147067382e+02 &  1.48413159102577e+02 & 5.12342382705810e-05 \\ 
\hline
22 & 1.48413156400487e+02 &  1.48413159102577e+02 & 1.20351947714425e-05 \\ 
\hline
23 & 1.48413158521648e+02 &  1.48413159102577e+02 & 2.70208917640957e-06 \\ 
\hline
24 & 1.48413158982770e+02 &  1.48413159102577e+02 & 5.80928826821037e-07 \\ 
\hline
25 & 1.48413159078837e+02 &  1.48413159102577e+02 & 1.19806998100103e-07 \\ 
\hline
26 & 1.48413159098050e+02 &  1.48413159102577e+02 & 2.37399433444807e-08 \\ 
\hline
27 & 1.48413159101745e+02 &  1.48413159102577e+02 & 4.52652670901443e-09 \\ 
\hline
28 & 1.48413159102429e+02 &  1.48413159102577e+02 & 8.31647639643052e-10 \\ 
\hline
29 & 1.48413159102551e+02 &  1.48413159102577e+02 & 1.47423406815506e-10 \\ 
\hline
30 & 1.48413159102572e+02 &  1.48413159102577e+02 & 2.52384779741988e-11 \\ 
\hline
31 & 1.48413159102576e+02 &  1.48413159102577e+02 & 4.17799128626939e-12 \\ 
\hline
32 & 1.48413159102576e+02 &  1.48413159102577e+02 & 6.53699316899292e-13 \\ 
\hline
33 & 1.48413159102577e+02 &  1.48413159102577e+02 & 8.52651282912120e-14 \\ 
\hline
34 & 1.48413159102577e+02 &  1.48413159102577e+02 & 0.00000000000000e+00 \\ 
\hline
\end{tabular}
\caption{Convergência de $e^x$ utilizando série de Taylor}
\label{table:erdos}
\end{table}

		Utilizando a série de Taylor foi percebido que o método converge após 24
		iterações.A complexidade deste método encontra-se no cáculo de
		\emph{n!}.  Podemos observar na Figura~\ref{eq:phi} a representação
		gráfica da convegência entre a série e $e^5$. 

		\begin{figure}[H] \centering
			\includegraphics[scale=.75]{ex5.png} 
		\caption{Convergência de $e^5$ em 40 iterações}
		\label{ex5} 
		\end{figure}

		Utilizamos o Algoritmo exponential implementado como mostrado a seguir,
		que retorna uma matriz de forma que cada coluna representa uma iteração
		e a primeira linha o valor calculado, a segunda o valor de $e^5$ e a
		tereceira linha é utilizada para guardar a diferença entre $e^5$ o valor
		calculado na iteração.

		\begin{lstlisting}
		function [ex] = exponential(x, iteration)
			ex2 = e^x;
			ex(1,1) = 1;
			ex(2,1) = ex2;
			ex(3,1) = ex2 - ex(1,1);
			for k=2:iteration
				ex(1,k) = ex(1,k-1) + (power(x,k-1) / factorial(k-1));
				ex(2,k) = ex2;
				ex(3,k) = ex2 - ex(1,k);
			end
		end
		\end{lstlisting}

	\subsection{Erdős-Borwein}

		A constante de Erdős-Borwein é a soma dos inversos multiplicativos dos
		números de Mersenne, e por definição é:

		\begin{equation}
			E = \displaystyle\sum_{n=1}^{\infty} \frac{1}{2^n-1} \approx 1.606695152415291763\dots
		\end{equation}

		Executando no \emph{octave}, obtemos os seguintes resultados:

		\begin{table}[H]
\centering 
\begin{tabular}{|c|c|c|c|}
\hline 
Iteração & Aproximação & Erro \\ 
\hline 
1 & 1.00000000000000e+00 &  6.06695152415292e-01 \\ 
\hline
2 & 1.33333333333333e+00 &  6.06695152415292e-01 \\ 
\hline
3 & 1.47619047619048e+00 &  2.73361819081958e-01 \\ 
\hline
4 & 1.54285714285714e+00 &  1.30504676224816e-01 \\ 
\hline
5 & 1.57511520737327e+00 &  6.38380095581490e-02 \\ 
\hline
6 & 1.59098822324629e+00 &  3.15799450420200e-02 \\ 
\hline
7 & 1.59886223899432e+00 &  1.57069291690042e-02 \\ 
\hline
8 & 1.60278380762177e+00 &  7.83291342097270e-03 \\ 
\hline
9 & 1.60474075478420e+00 &  3.91134479352173e-03 \\ 
\hline
10 & 1.60571827189075e+00 &  1.95439763109517e-03 \\ 
\hline
11 & 1.60620679167580e+00 &  9.76880524545809e-04 \\ 
\hline
12 & 1.60645099192000e+00 &  4.88360739494542e-04 \\ 
\hline
13 & 1.60657307713548e+00 &  2.44160495294299e-04 \\ 
\hline
14 & 1.60663411601725e+00 &  1.22075279813894e-04 \\ 
\hline
15 & 1.60666463452672e+00 &  6.10363980462214e-05 \\ 
\hline
16 & 1.60667989354862e+00 &  3.05178885702251e-05 \\ 
\hline
17 & 1.60668752300136e+00 &  1.52588666735287e-05 \\ 
\hline
18 & 1.60669133771318e+00 &  7.62941393417371e-06 \\ 
\hline
19 & 1.60669324506545e+00 &  3.81470211663348e-06 \\ 
\hline
20 & 1.60669419874067e+00 &  1.90734984584218e-06 \\ 
\hline
21 & 1.60669467557806e+00 &  9.53674619941225e-07 \\ 
\hline
22 & 1.60669491399669e+00 &  4.76837234364424e-07 \\ 
\hline
23 & 1.60669503320600e+00 &  2.38418598419443e-07 \\ 
\hline
24 & 1.60669509281065e+00 &  1.19209294657807e-07 \\ 
\hline
25 & 1.60669512261297e+00 &  5.96046463297029e-08 \\ 
\hline
26 & 1.60669513751413e+00 &  2.98023230538291e-08 \\ 
\hline
27 & 1.60669514496471e+00 &  1.49011616379369e-08 \\ 
\hline
28 & 1.60669514869000e+00 &  7.45058104101304e-09 \\ 
\hline
29 & 1.60669515055265e+00 &  3.72529074255112e-09 \\ 
\hline
30 & 1.60669515148397e+00 &  1.86264559332017e-09 \\ 
\hline
31 & 1.60669515194963e+00 &  9.31323018704688e-10 \\ 
\hline
32 & 1.60669515218246e+00 &  4.65661731396949e-10 \\ 
\hline
33 & 1.60669515229888e+00 &  2.32831087743079e-10 \\ 
\hline
34 & 1.60669515235708e+00 &  1.16415765916145e-10 \\ 
\hline
35 & 1.60669515238619e+00 &  5.82081050026773e-11 \\ 
\hline
36 & 1.60669515240074e+00 &  2.91042745459436e-11 \\ 
\hline
37 & 1.60669515240802e+00 &  1.45523593175767e-11 \\ 
\hline
38 & 1.60669515241165e+00 &  7.27640170339328e-12 \\ 
\hline
39 & 1.60669515241347e+00 &  3.63842289630156e-12 \\ 
\hline
40 & 1.60669515241438e+00 &  1.81943349275571e-12 \\ 
\hline
41 & 1.60669515241484e+00 &  9.09938790982778e-13 \\ 
\hline
42 & 1.60669515241506e+00 &  4.55191440096314e-13 \\ 
\hline
43 & 1.60669515241518e+00 &  2.27817764653082e-13 \\ 
\hline
44 & 1.60669515241523e+00 &  1.14130926931466e-13 \\ 
\hline
45 & 1.60669515241526e+00 &  5.72875080706581e-14 \\ 
\hline
\end{tabular}
\caption{Convergência de Erdős-Borwein}
\label{table:erdos}
\end{table}

		Podemos ver o número se estabilizando no seguinte gráfico,
		construído a partir dos resultados da tabela.

		\begin{figure}[H]
			\centering
			\includegraphics[width=100mm]{erdos.png}
			\caption{Convergência do erdős-borwein em 50 iterações}
			\label{erdos_graphic}
		\end{figure}

		A definição de $E$ foi codificada da seguinte maneira:

		\begin{lstlisting}

		function [er] = erdos(iteration)
			er(1) = 1;
			for i=2:iteration
				er(i) = er(i-1) + (1 / ((2^i)-1));
			end
		end

		\end{lstlisting}
