\section{Métodos Iterativos}

Um \textbf{método iterativo} é um procedimento matemático para resolução de
equações que gera uma sequência de aproximações, que servem como solução para
uma classe de problemas. Uma implementação específica de um método iterativo,
incluindo sua condição de parada, é um algoritmo. Um método iterativo é
considerado \textbf{convergente} se a sequência de valores geradas converge para
uma dada aproximação inicial.

Para uma definição mais teórica, o seguinte autor define:

\begin{quotation}

``Um método iterativo consiste em repetir uma determinada operação um certo número
de vezes até que nos seja fornecida uma aproximação, que satisfaça as condições
do problema e, para tal, a sequência de valores deve ser
convergente.''\cite{batista2014metodos}

\end{quotation}

Nos problemas do tipo encontre a raíz da equação, um método iterativo usa
uma suposição inicial para gerar sucessivas aproximações à uma solução. Em
Contraste, \textbf{métodos diretos} tentam resolver o problema em uma sequência
\emph{finita} de operações.

Um método iterativo é formado por quatro partes:~\cite{claudio2000calculo}

\begin{enumerate}[a)]
	\item Estimativa inicial: uma ou mais aproximações para a raiz desejada.
    \item Atualização: uma fórmula que atualize a solução aproximada.
    \item Critério de parada: uma forma de estabelecer quando parar o processo iterativo em qualquer caso.
    \item Estimador de exatidão: está associado ao critério de parada e provê uma estimativa do erro cometido.
\end{enumerate}

\newpage

\section{Exercícios}
%PI = http://mathforum.org/library/drmath/view/65244.html
Nessa seção vamos encontrar números irracionais conhecidos utilizando métodos iterativos.

\subsection{Número de Ouro}

	O número de ouro, também conhecido como $\phi$, é o resultado da seguinte
	equação:

	\begin{equation}
		x^2-x-1
	\end{equation}

	Existem algums algoritmos para realizar a aproximação neste artigo, sendo
	duas destas formas  abordadas. A primeira utilizando frações continuas e o
	seguundo calculando as raizaes da equação utiizando método de newton.

\subsubsection{Frações Continuadas}

	Frações conitinuadas são formas de representar números reais de tal forma
	que a expressão básica tem o seguinte formato:
	\begin{equation}\label{eq:phi}
		a_0 + \frac{b_0}{a_1 + \frac{b_1}{a_n + \dots}}
	\end{equation}

	Para calcular o $\phi$ devemos substituir \emph{a} e \emph{b} por 1. Na
	Fração~\ref{eq:phi}.

\begin{table}[H]
	\centering
	\begin{tabular}{|c|c|c|c|}

		\hline
		Iteração & $x$ & Iteração & $x$ \\
		\hline
		1 & 2.00000000000000000000 & 26 & 1.61803398873830306393 \\
		\hline
		2 & 1.50000000000000000000 & 27 & 1.61803398875432247195 \\
		\hline
		3 & 1.66666666666666651864 & 28 & 1.61803398874820381081 \\
		\hline
		4 & 1.60000000000000008882 & 29 & 1.61803398875054060824 \\
		\hline
		5 & 1.62500000000000000000 & 30 & 1.61803398874964821097 \\
		\hline
		6 & 1.61538461538461541878 & 31 & 1.61803398874998904944 \\
		\hline
		7 & 1.61904761904761906877 & 32 & 1.61803398874985893130 \\
		\hline
		8 & 1.61764705882352943789 & 33 & 1.61803398874990866929 \\
		\hline
		9 & 1.61818181818181816567 & 34 & 1.61803398874988957346 \\
		\hline
		10 & 1.61797752808988759554 & 35 & 1.61803398874989690093 \\
		\hline
		11 & 1.61805555555555558023 & 36 & 1.61803398874989401435 \\
		\hline
		12 & 1.61802575107296142676 & 37 & 1.61803398874989512457 \\
		\hline
		13 & 1.61803713527851456000 & 38 & 1.61803398874989490253 \\
		\hline
		14 & 1.61803278688524576623 & 39 & 1.61803398874989490253 \\
		\hline
		15 & 1.61803444782168193150 & 40 & 1.61803398874989490253 \\
		\hline
		16 & 1.61803381340012508716 & 41 & 1.61803398874989490253 \\
		\hline
		17 & 1.61803405572755432118 & 42 & 1.61803398874989490253 \\
		\hline
		18 & 1.61803396316670644595 & 43 & 1.61803398874989490253 \\
		\hline
		19 & 1.61803399852180351814 & 44 & 1.61803398874989490253 \\
		\hline
		20 & 1.61803398501735795634 & 45 & 1.61803398874989490253 \\
		\hline
		21 & 1.61803399017559712547 & 46 & 1.61803398874989490253 \\
		\hline
		22 & 1.61803398820532517988 & 47 & 1.61803398874989490253 \\
		\hline
		23 & 1.61803398895790184753 & 48 & 1.61803398874989490253 \\
		\hline
		24 & 1.61803398867044334608 & 49 & 1.61803398874989490253 \\
		\hline
		25 & 1.61803398878024262686 & 50 & 1.61803398874989490253 \\
		\hline

	\end{tabular}
	\label{golden_fraction}
	\caption{Convergência do número de ouro pelo método de frações continuadas}
\end{table}

Como fica ilustrado na tabela, o número já se estabiliza antes mesmo de
chegarmos à 50 iterações.

\begin{figure}[H]
    \centering
    \includegraphics[width=100mm]{golden_frac.png}
    \caption{Convergência do número de ouro pelo método de frações continuadas}
    \label{golden_frac}
\end{figure}


O método iterativo utilizado foi descrito pelo seguinte código:

\begin{lstlisting}

function [x] =  phi_frac(iteration=1)
	aux = 1;
	for i= 1:iteration
		aux = double(1 + 1/aux);
		x(i) = aux;
	end
end

\end{lstlisting}

\subsubsection{Método de Newton}

A ideia do método de Newton é a partir de um valor inicial arbitrário, então a
função é aproximada por sua reta tangente. O $x$ que intercepta a reta e a
função é computado, e ele será uma melhor aproximação que o chute inicial. O
método, então, pode ser iterado.

Analizando novamente o número de ouro, mas com o método de newton, um número
muito menor de iterações é observado.

\begin{table}[H]
	\centering
	\begin{tabular}{|c|c|c|}
		\hline
		Iteração & $x$ & erro \\
		\hline
		1 & 1.66666666666666674068 & 0.33333333333333325932 \\
		\hline
		2 & 1.61904761904761906877 & 0.04761904761904767192 \\
		\hline
		3 & 1.61803444782168193150 & 0.00101317122593713727 \\
		\hline
		4 & 1.61803398874998904944 & 0.00000045907169288206 \\
		\hline
		5 & 1.61803398874989490253 & 0.00000000000009414691 \\
		\hline
		6 & 1.61803398874989490253 & 0.00000000000000000000 \\
		\hline
	 \end{tabular}
	\label{golden_newton}
	\caption{Convergência do número de ouro pelo método de newton}
\end{table}

Isso pode ser observado no gráfico abaixo.

\begin{figure}[H]
    \centering
    \includegraphics[width=100mm]{golden_newton.png}
    \caption{Convergência do número de ouro pelo método de newton}
    \label{golden_newton}
\end{figure}

\newpage

O método iterativo utilizado foi descrito pelo seguinte código:

\begin{lstlisting}

function [x, ex] = newton( f, df, x0, tol, nmax)
	f = inline(f);
	df = inline(df);
	x(1) = double(x0 - (f(x0)/df(x0)));
	ex(1) = abs(x(1)-x0);
	k = 2;
	while k <= nmax && ex(k-1) > tol
		 x(k) = double(double(x(k-1)) - double((f(x(k-1))/df(x(k-1)))));
		 ex(k) = abs(x(k)-x(k-1));
		 k = k+1;
	end
end

\end{lstlisting}

\subsection{Pi($\pi$)}

\subsubsection{Método Utilizando Funções Trigonométricas}

Encontramos em um
\href{http://mathforum.org/library/drmath/view/65244.html}{fórum de matemática}
um método iterativo que calcula $ \pi $ de uma forma aparentemente mais simples,
apesar de sua complexidade estar escondida na função \emph{sin}. O método está
descrito a seguir:

\begin{equation}
\label{magic_equation}
P(n+1) = P(n) + sin(P(n))
\end{equation}

$P(n)$ seria a aproximação de $\pi$ na iteração $n$. Esse método consegue
convergir para $\pi$ com um número muito baixo de iterações.

\begin{table}[H]
	\centering
	\begin{tabular}{|c|}
    	\hline
		$P(n)$ \\
		\hline
		3.14112000805986735230135309393517673015594482421875 \\
		\hline
		3.14159265357219563696844488731585443019866943359375 \\
		\hline
		3.14159265358979311599796346854418516159057617187500 \\
		\hline
		3.14159265358979311599796346854418516159057617187500 \\
		\hline
		3.14159265358979311599796346854418516159057617187500 \\
		\hline
		\hline
		$\pi$ \\
		\hline
		3.14159265358979311599796346854418516159057617187500 \\
		\hline
		\hline
		$P(n) - \pi$  Para visualizar a diferença \\
		\hline
		-4.72645529925763696610374609008431434631347656250000 \\
		\hline
		-1.75974790295185812283307313919067382812500000000000 \\
		\hline
		0.00000000000000000000000000000000000000000000000000 \\
		\hline
		0.00000000000000000000000000000000000000000000000000 \\
		\hline
		0.00000000000000000000000000000000000000000000000000 \\
		\hline
	\end{tabular}
	\label{pi_magic}
	\caption{Convergência do $\pi$}
\end{table}

O seguinte gráfico foi gerado com a análise dos resultos:

\begin{figure}[H]
    \centering
    \includegraphics[width=100mm]{pi_magic.png}
    \caption{Convergência do $\pi$}
    \label{pi_magic}
\end{figure}

O método iterativo utilizado foi descrito pelo seguinte código:

\begin{lstlisting}

function [pif, pi_vec] = pi_it(iteration)
	pif(1) = 3 + sin(3);
	pi_vec(1) = pi
	for i = 2:iteration
		pif(i) = pif(i-1) + sin(pif(i-1));
		pi_vec(i) = pi;
		aux = pif(i);
	end
end

\end{lstlisting}

\subsection{$e^x$}

	A função $e^x$ é um função exponencial cuja base é o número de Euler,
	também conhecida como função exponencial natural.

	Para o calculo da da função utilizamos a seguinte série de Taylor
	
	\begin{equation}
		\sum_{n=1}^{n} = \frac{x^n}{n!}
	\end{equation}

	Para exemplificar a foram realizadas 40 iteração para a seguinte função $e^5$, e obtivemos os seguintes resultaodos:

\begin{table}[H]
	\centering
	\begin{tabular}{|c|c|c|c|c|}
    	\hline
		Iteração & Valor Iteração & $e^5$ & Diferença \\
		\hline
		1 & 1.000000 & 148.413159 & 147.413159\\
		\hline
		2 & 6.000000 & 148.413159 & 142.413159\\
		\hline
		3 & 18.500000 & 148.413159 & 129.913159\\
		\hline
		4 & 39.333333 & 148.413159 & 109.079826\\
		\hline
		5 & 65.375000 & 148.413159 & 83.038159\\
		\hline
		6 & 91.416667 & 148.413159 & 56.996492\\
		\hline
		7 & 113.118056 & 148.413159 & 35.295104\\
		\hline
		8 & 128.619048 & 148.413159 & 19.794111\\
		\hline
		9 & 138.307168 & 148.413159 & 10.105991\\
		\hline
		10 & 143.689457 & 148.413159 & 4.723703\\
		\hline
		11 & 146.380601 & 148.413159 & 2.032558\\
		\hline
		12 & 147.603849 & 148.413159 & 0.809311\\
		\hline
		13 & 148.113535 & 148.413159 & 0.299624\\
		\hline
		14 & 148.309568 & 148.413159 & 0.103591\\
		\hline
		15 & 148.379580 & 148.413159 & 0.033579\\
		\hline
		16 & 148.402917 & 148.413159 & 0.010242\\
		\hline
		17 & 148.410210 & 148.413159 & 0.002949\\
		\hline
		18 & 148.412355 & 148.413159 & 0.000804\\
		\hline
		19 & 148.412951 & 148.413159 & 0.000208\\
		\hline
		20 & 148.413108 & 148.413159 & 0.000051\\
		\hline
		21 & 148.413147 & 148.413159 & 0.000012\\
		\hline
		22 & 148.413156 & 148.413159 & 0.000003\\
		\hline
		23 & 148.413159 & 148.413159 & 0.000001\\
		\hline
		24 & 148.413159 & 148.413159 & 0.000000\\
		\hline
		25 & 148.413159 & 148.413159 & 0.000000\\
		\hline
		26 & 148.413159 & 148.413159 & 0.000000\\
		\hline
		27 & 148.413159 & 148.413159 & 0.000000\\
		\hline
		28 & 148.413159 & 148.413159 & 0.000000\\
		\hline
		29 & 148.413159 & 148.413159 & 0.000000\\
		\hline
		30 & 148.413159 & 148.413159 & 0.000000\\
		\hline
		31 & 148.413159 & 148.413159 & 0.000000\\
		\hline
		32 & 148.413159 & 148.413159 & 0.000000\\
		\hline
		33 & 148.413159 & 148.413159 & 0.000000\\
		\hline
		34 & 148.413159 & 148.413159 & 0.000000\\
		\hline
		35 & 148.413159 & 148.413159 & 0.000000\\
		\hline
		36 & 148.413159 & 148.413159 & 0.000000\\
		\hline
		37 & 148.413159 & 148.413159 & 0.000000\\
		\hline
		38 & 148.413159 & 148.413159 & 0.000000\\
		\hline
		39 & 148.413159 & 148.413159 & 0.000000\\
		\hline
		40 & 148.413159 & 148.413159 & 0.000000\\
		\hline
	\end{tabular}
	\label{ex_table}
	\caption{Convergência de $e^5$}
\end{table}

Utilizando a série de Taylor foi percebido que o método converge após 24
iterações.A complexidade deste método encontra-se no cáculo de \emph{n!}.
Podemos observar na Figura~\ref{eq:phi} a representação gráfica da convegência
entre a série e $e^5$. 

\begin{figure}[H] \centering
	\includegraphics[scale=.75]{ex5.png} 
\caption{Convergência de $e^5$ em 40 iterações}
\label{ex5} 
\end{figure}

Utilizamos o Algoritmo exponential implementado como mostrado a seguir,
que retorna uma matriz de forma que cada coluna representa uma iteração e
a primeira linha o valor calculado, a segunda o valor de $e^5$ e a
tereceira linha é utilizada para guardar a diferença entre $e^5$ o valor
calculado na iteração.

\begin{lstlisting}
function [ex] = exponential(x, iteration)
	ex2 = e^x;
	ex(1,1) = 1;
	ex(2,1) = ex2;
	ex(3,1) = ex2 - ex(1,1);
	for k=2:iteration
		ex(1,k) = ex(1,k-1) + (power(x,k-1) / factorial(k-1));
		ex(2,k) = ex2;
		ex(3,k) = ex2 - ex(1,k);
	end
end
\end{lstlisting}

\subsection{Erdős-Borwein}

A constante de Erdős-Borwein é a soma dos inversos multiplicativos dos números
de Mersenne, e por definição é:

\begin{equation}
	E = \displaystyle\sum_{n=1}^{\infty} \frac{1}{2^n-1} \approx 1.606695152415291763\dots
\end{equation}

Executando no \emph{octave}, obtemos os seguintes resultados:

\begin{table}[H]
	\centering
	\begin{tabular}{|c|c|c|c|}
    	\hline
		Iteração & $E$ & Iteração & $E$ \\
    	\hline
		1 & 1.0000000000 & 19 & 1.6066932451 \\
    	\hline
		2 & 1.3333333333 & 20 & 1.6066941987 \\
    	\hline
		3 & 1.4761904762 & 21 & 1.6066946756 \\
    	\hline
		4 & 1.5428571429 & 22 & 1.6066949140 \\
    	\hline
		5 & 1.5751152074 & 23 & 1.6066950332 \\
    	\hline
		6 & 1.5909882232 & 24 & 1.6066950928 \\
    	\hline
		7 & 1.5988622390 & 25 & 1.6066951226 \\
    	\hline
		8 & 1.6027838076 & 26 & 1.6066951375 \\
    	\hline
		9 & 1.6047407548 & 27 & 1.6066951450 \\
    	\hline
		10 & 1.6057182719 & 28 & 1.6066951487 \\
    	\hline
		11 & 1.6062067917 & 29 & 1.6066951506 \\
    	\hline
		12 & 1.6064509919 & 30 & 1.6066951515 \\
    	\hline
		13 & 1.6065730771 & 31 & 1.6066951519 \\
    	\hline
		14 & 1.6066341160 & 32 & 1.6066951522 \\
    	\hline
		15 & 1.6066646345 & 33 & 1.6066951523 \\
    	\hline
		16 & 1.6066798935 & 34 & 1.6066951524 \\
    	\hline
		17 & 1.6066875230 & 35 & 1.6066951524 \\
    	\hline
		18 & 1.6066913377 & 36 & 1.6066951524 \\
    	\hline
	\end{tabular}
	\label{erdos_table}
	\caption{Convergência do $E$}
\end{table}

Podemos ver o número se estabilizando no seguinte gráfico, construído a partir
dos resultados da tabela.

\begin{figure}[H]
    \centering
    \includegraphics[width=100mm]{erdos.png}
    \caption{Convergência do erdős-borwein em 50 iterações}
	\label{erdos_graphic}
\end{figure}

A definição de $E$ foi codificada da seguinte maneira:

\begin{lstlisting}

function [er] = erdos(iteration)
	er(1) = 1;
	for i=2:iteration
		er(i) = er(i-1) + (1 / ((2^i)-1));
	end
end

\end{lstlisting}
