\section{Computação Cientifica no Brasil e no mundo}
\label{sec:computacao-cientifica}

	No brasil, existem alguns institutos dedicados a computação científica, como
	por exemplo o LNCC (Laboratório Nacional de Computação Científica) ou a
	SBMAC (Sociedade Brasileira de Matemática Aplicada e Computacional). No
	mundo, os mais conhecidos são: SCG (Symbolic Computation Group), NAG
	(Numerical Algorithms Group), entre outros.

	Aprofundando mais nas áreas de pesquisa do LNCC, já podemos ter uma idéia de
	como é grande a abrangência da computação científica e a complexidade de
	seus estudos.

	\begin{itemize}

		\item \textbf{Modelagem Computacional} é uma área de conhecimento
			multidisciplinar que lida com pesquisa e desenvolvimento de métodos
			para compreensão e solução de problemas com embasamento na
			fenomenologia, na matemática e na computação, em áreas tão
			abrangentes quanto as engenharias e as ciências exatas, econômicas,
			biológicas, humanas e ambientais, dentre outras.

		\item \textbf{Métodos Numéricos} - Essa linha de pesquisa é voltada ao
			desenvolvimento e análise matemática de técnicas e algoritmos
			numéricos para a simulação computacional de fenômenos complexos.
			Mais especificamente, a pesquisa caracteriza-se por uma ou mais das
			seguintes etapas:
			\begin{enumerate}

				\item Análise Numérica e Adaptatividade

				\item Meta-Heurísticas

				\item Métodos de Elementos Finitos

			\end{enumerate}

		\item \textbf{Sistemas, Controle e Sinais} - A Teoria de Sistemas,
			Controle e Sinais tem como objetivo o estudo do comportamento de
			sistemas dinâmicos, visando a atingir determinados padrões de
			referências do estado ou da saída do sistema. É uma área
			intrinsecamente interdisciplinar. Como as técnicas oriundas dessa
			área dependem, entre outros, do modelo utilizado, do conjunto de
			informações disponíveis sobre os parâmetros e variáveis do modelo e
			do tipo de incertezas consideradas, ela é uma área de pesquisa
			bastante abrangente.  Tem aplicações nas mais variadas áreas das
			ciências e engenharias, incluindo ecologia, economia e robótica, e
			tem sido reconhecida como fundamental no desenvolvimento de novas
			áreas, da nanotecnologia à regulação de células.

		\item \textbf{Computação} - Essa linha de pesquisa é focada nos desafios
			e paradigmas que surgem na computação como um todo e especificamente
			na computação massivamente paralela e distribuída, na computação
			quântica, na visualização científica e nos ambientes colaborativos
			de realidade virtual e aumentada e de interconectividade oferecida
			por redes de computadores, no desenvolvimento de banco de dados de
			maneira a impactar a pesquisa e o desenvolvimento de modelos,
			métodos e algoritmos robustos e altamente eficientes, do ponto de
			vista computacional.

		\item A \textbf{Biologia Computacional} tem como objetivo integrar
			conhecimentos da Ciência da Computação, Matemática Aplicada e
			Estatística com a Biologia para o desenvolvimento de modelos e
			ferramentas computacionais que permitam analisar dados e fenômenos
			biológicos. 

		\item \textbf{Petróleo, Água e Gás} - Modelagem determinística e
			estocástica e a simulação computacional de escoamentos miscíveis e
			imiscíveis incorporando acoplamento geomecânico em reservatórios de
			petróleo altamente heterogêneos, na avaliação da capacidade de carga
			residual computacional de escoamentos miscíveis e imiscíveis de
			dutos corroídos, na análise de dutos em zig-zag e de tensões em
			armaduras de risers flexíveis e na visualização de plataformas de
			exploração de campos de petróleo ao largo da costa (offshore).

		\item \textbf{Medicina Assistida por Computação Científica} - O avanço
			da Computação Científica gera, na área médica, grandes e profundas
			modificações ao permitir:
			\begin{itemize}

				\item a síntese do diagnóstico por imagem que, acoplada à
					modelagem e simulação, permite o desenvolvimento de novas
					técnicas terapêuticas em tempo real, para melhorar
					procedimentos e tratamentos médicos;

				\item o desenvolvimento de modelos e simuladores precisos dos
					diversos sistemas do corpo humano e da sua interrelação, que
					integram anatomia, fisiologia, propriedades biomecânicas,
					biologia celular e bioquímica, para aplicações terapêuticas,
					de pesquisa, de formação e de treinamento de recursos
					humanos;

				\item o desenvolvimento de um “corpo virtual” para cada
					paciente, de maneira a servir como um repositório para
					diagnóstico, patologias e outras informações médicas sobre o
					paciente.  Esse “corpo virtual” permite aumentar a
					comunicação entre paciente e médico e fornece referência
					para exames, patologias e mudanças que acontecem com o
					passar do tempo;

				\item a utilização de modelos e simuladores de alta precisão,
					para planejamento cirúrgico, treinamento e credenciamento
					médico. Os simuladores permitem verdadeira interação do
					usuário com órgãos humanos simulados, com propriedades
					físicas e fisiológicas realísticas, úteis para educação e
					pesquisa e desenvolvimento de aplicações médicas.

			\end{itemize}

	\end{itemize}
