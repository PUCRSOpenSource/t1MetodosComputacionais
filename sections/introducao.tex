\section*{Introdução}

	Dentro do escopo da disciplina de \emph{Métodos Computacionais} o trabalho
	pode ser resumido da seguinte maneira: deve ser realizada um estudo sobre a
	área de computação científica no Brasil e no mundo, este estudo deve ser
	apresentado de forma que seja possível idenficar as áreas dentro da
	computação científica.

	Ainda no escopo do primeiro trabalho da disciplina é necessaŕio apresentar o
	que são métodos iterativos e como eles se comportam. Também é preciso
	mostrar como se calcula, utilizando os métodos iterativos,  os seguintes
	números irracionais:

	\begin{itemize}
		\item A constante áurea,
		\item A constante de Euler,
		\item A função $e^x$,
		\item O número pi,
		\item A constante de Erdős–Borwein.
	\end{itemize}

	Este artigo está organizado da seguinte maneira: na
	Seção~\ref{sec:computacao-cientifica} será abordado a computação científica
	e sua importância. Na seção~\ref{sec:metodos-iterativos} será discutido
	métodos iterativos e serão apresentados cálculos de números irracionais
	utlizando esses métodos. Na seção~\ref{sec:conclusao} serão apresentadas as
	conclusões que foram alcançadas com os estudos realizados.
