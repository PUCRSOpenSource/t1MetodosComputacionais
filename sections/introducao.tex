\section*{Introdução}

	Dentro do escopo da disciplina de \emph{Métodos Computacionais} o trabalho
	pode ser resumido da seguinte maneira: deve ser realizada um estudo sobre a
	área de computação científica no Brasil e no mundo, este estudo deve ser
	apresentado de forma que seja possível se idenficar as áreas que são
	tratadas na área e também quas as linhas de pesquisa que podem ser
	encontradas.

	Ainda no escopo do primeiro trabalho da disciplina é necessaŕio apresentar o
	que são e como se comportam métodos iterativos de forma que devem ser
	calculdaos de forma iterativa os seguintes números irracionais:

	\begin{itemize}
		\item A constante áurea,
		\item A constante de Euler,
		\item A função $e^x$,
		\item O número pi,
		\item A constante de Erdős–Borwein.
	\end{itemize}

	Este artigo irá está organizado da seguinte maneira: na primeira seção será
	abordado a computação científica e sua importância, na seção dois será
	discutido métodos iterativo e serão apresentados calculos de números irracionais utlizando
	este(s) métodos.Na seção três serão apresentadas a conclusões que
	foram alcançadas com os estudos realizados.
